\chapter*{Synopsis}


This project investigates the behaviour and mechanics of natural swarm systems such as social insects, schools of fish and flocks of birds, and applies this to a group of robots. Purely through robot-robot interactions and without any direction from a supervisor, this swarm of robots is capable of quick self-organisation in order to aggregate and move as a cohesive unit. 

Imitating nature in this manner is useful. Owing to its decentralised nature, a swarm system possesses a few desirable traits such as adaptation, fault tolerance and scalability. These properties can be useful if applied to an artificial system, leading to more robust engineering designs.