\chapter{Literature Review}
\label{chap:LiteratureReview}

In this chapter, an analysis of previous work completed in the field of swarm robotics is provided, as well as an outline of the theoretical concepts concerning the techniques applied during this project.

\section{Biologically Inspired Computation and Engineering}

Nature has provided examples of outwardly complex biological systems which are often efficient, fluid
and resilient to partial breakdowns. Colonies of ants are able to forage for food and build large, complex
structures. Fireflies are able to synchronize their flashing with one another. Flocks of birds and schools of fish exhibit fluid
and efficient group movement. In the majority of cases, nature has achieved these despite utilizing very little to
no communication between individual creatures and in the absence of a higher level director or supervisor.
The animals react only to environmental stimuli, either the strength or type of pheromones detected in the environment in the case of
ants (termed \textsl{stigmergy}) or the positions of other individuals in the case of fish and birds. For example, it has been shown that over time, the velocity of a flock of birds converges to a state where each individual has equal velocity. \cite{cucker-smale}

This is defined as \textsl{emergent behaviour}, the rise of previously unpredicted and complex behaviour through the interaction of simple rules.

Researchers have attempted to replicate this in an artificial system via various means. For example, Bojinov et. al (2000) demonstrated how simple, local sensory rules can produce complex and useful structures in metamorphic robots. \cite{Bojinov2000}  Metamorphic, or modular, robots are a relatively new concept which can be described as "a robot that is composed of modules that are all identical". \cite{Yim} In their paper, Bojinov et. al explored the use of purely \textit{local} rules and interactions to develop control algorithms which enabled a Proteo robot to accomplish some otherwise difficult and complex tasks. Such tasks include forming three dimensional structures, locomotion and dynamic adaptation.

Previous work on this front has utilised an exact, a-priori definition of the configuration of the robot, calculated via algorithms that use heuristics. \cite{metricsrob} These algorithms all require knowledge of the problem, the environment and the desired shape before hand, which in a real world setting is not always feasible.  What makes the work of Bojinov et. al interesting is that their algorithm does not aim to produce a pre-defined shape, rather it aims to produce a shape with the desired \textsl{properties} needed to solve the problem at hand. In this way, the algorithm produced is robust to unknowns, noise or changes in the environment, in a similar manner to how animals can adapt to changes in their environment.

From an engineering perspective, mimicking these systems could provide better solutions to a multitude
of problems across various industries. One way that is gaining major interest from academics and the
industry alike is the application of swarm robotics. Swarm robotics is a relatively new field of multi-
robotics, in which the aim is to co-ordinate a large number of robots in a decentralized manner, similar to
the natural systems mentioned previously. In order to carry out this project, a thorough understanding of
the concepts and mechanisms that underpin swarm robotics systems will need to be achieved, as well as
a strong working understanding of various tools used in their implementation such as simulation software
and higher level programming.

\section{Artificial Intelligence}
Something about aim

\section{Collective and Swarm Intelligence}
Aims and objectives are different, somehow
