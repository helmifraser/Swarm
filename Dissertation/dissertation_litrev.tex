\chapter{Literature Review}
\label{chap:LiteratureReview}

Nature has provided examples of outwardly complex biological systems which are often efficient, fluid
and resilient to partial breakdowns. Colonies of ants are able to forage for food and build complex
structures. Fireflies are able to synchronize their flashing. Flocks of birds and schools of fish exhibit fluid
and efficient group movement. In the majority of cases, nature has achieved these utilizing very little to
no communication between individual creatures and in the absence of a higher level director or supervisor.
The animals react only to environmental stimuli, either the strength or type of pheromones in the case of
ants or the positions of other individuals in the case of fish and birds.

This is defined as emergent behavior, the rise of previously unpredicted, complex behavior through the
interaction of simple rules.

From an engineering perspective, mimicking these systems could provide better solutions to a multitude
of problems across various industries. One way that is gaining major interest from academics and the
industry alike is the application of swarm robotics. Swarm robotics is a relatively new field of multi-
robotics, in which the aim is to co-ordinate a large number of robots in a decentralized manner, similar to
the natural systems mentioned previously. In order to carry out this project, a thorough understanding of
the concepts and mechanisms that underpin swarm robotics systems will need to be achieved, as well as
a strong working understanding of various tools used in their implementation such as simulation software
and higher level programming.

\section{Artificial Intelligence}
Something about aim

\section{Swarm Inte}
Aims and objectives are different, somehow

\section{Relevance}
Swarm shit is cool