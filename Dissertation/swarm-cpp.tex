\begin{lstlisting}[language=C++, caption={swarm.cpp},label={lst:swarm-header}]
  #include "swarm.hpp"

  Swarm::Swarm() {

    compass = getCompass("compass");
    compass->enable(TIME_STEP);

    gps = getGPS("gps");
    gps->enable(TIME_STEP);

    std::string ps = "ps";
    std::string led = "led";

    for (unsigned int i = 0; i < 8; i++) {
      distanceSensors[i] = getDistanceSensor(ps.replace(2, 1, std::to_string(i)));
      distanceSensors[i]->enable(TIME_STEP);
      leds[i] = getLED(led.replace(3, 1, std::to_string(i)));
    }

    emitter = getEmitter("emitter");
    emitter->setRange(RANGE);
    receiver = getReceiver("receiver");
    receiver->enable(TIME_STEP);

    keyboard = getKeyboard();
    keyboard->enable(TIME_STEP);

    robot_name = getName();
    data = "";

    speed = {100, 50, 0};
    multiplier = 1;

    myData.received = false;
    myData.data[6] = 0;

    robots = 0;
    srand(time(0));
    choose = rand() % 3;
    aligned = false;

    for (int i = 0; i < 100; i++) {
      for (int j = 0; j < 100; j++) {
        matrix[i][j] = 0;
      }
    }

    for (int i = 0; i < 100; i++) {
      for (int j = 0; j < 100; j++) {
        srand(time(NULL) * i + j);
        matrix[i][j] = rand() % 15872567351;
      }
    }
  }

  /* this function waits for a keyboard input to specify which mode the simulation
   *    should run in
   */

  void Swarm::run() {
    std::cout << "1 for keyboard control, 2 for object avoidance, 3 for "
                 "flocking"
              << std::endl;
    std::cout << "Press 5 to stop the simulation. Not doing this will corrupt "
                 "the data output."
              << std::endl;
    while (step(TIME_STEP) != -1) {
      int decide = readKey();
      switch (decide) {
      case 49:
        std::cout << "Keyboard control mode (epuck2)" << std::endl;
        std::cout << "WASDQE keys for control" << std::endl;
        teleop();
        break;
      case 50:
        std::cout << "Object avoidance mode" << std::endl;
        objectDetectionMode();
        break;
      case 51:
        std::cout << "Flock" << std::endl;
        flock();
        break;
      }
    }
  }

  /*
    Flocking algorithm.
    This function provides the swarming behaviour.
    A detailed explanation of which can be found in the Implementation section.
   */

  void Swarm::flock() {

    // creates a robot specific .csv file for outputting data to
    std::string filename = printName() + "_data.csv";
    std::ofstream data(filename, std::ios::out);
    if (data.is_open()) {
      data << printName() << "\n";
      data << "x-value,z-value,robots,orientation\n";
    }

    // spins each robot to a random angle
    randomAlign();

    int quit = 0;
    int step_count = 0;

    // this loop runs infinitely until Webots throws an error or the escape key is
    // pressed
    while (step(TIME_STEP != -1) && quit != 1) {

      getReceiverData();
      processReceiverData(myData.orientationString);

      saveCompassValues();
      distanceCheck();

      int index = std::distance(
          signalStrength,
          std::max_element(signalStrength,
                           signalStrength +
                               sizeof(signalStrength) / sizeof(double)));

      std::array<double, 3> direction = {0, 0, 0};
      if (robots > ALIGN_THRESHOLD) {
        for (int k = 0; k < 3; k++) {
          direction[k] = emitterDirection[index + 1][k];
        }
      } else if (robots == 1) {
        for (int k = 0; k < 3; k++) {
          direction[k] = emitterDirection[index][k];
        }
      }

      double nearestNeighbour = computeVectorAngle(direction);

      /* simple finite state machine arbitration

         If the conditions are right to swarm, robot adjusts its heading.
         Else, move away from any obstacles.

       */

      if ((robots >= ALIGN_THRESHOLD) && (!left_obstacle) && (!right_obstacle) &&
          signalStrength[index] < 10) {
        setLEDs(0);
        adjust(nearestNeighbour);

      } else {
        setLEDs(1);
        objectDetection(1.0);
      }

      double robotsDouble = robots;

      /* Dynamically adaptable range.
         This adds a "gravitational" aspect to a large group, encouraging
         robots to aggregate.
       */

      if (robots >= 1) {
        emitter->setRange(RANGE * (robotsDouble + 1));
      }

      sendCurrentOrientation();
      step_count++;

      int quitKey = readKey();

      switch (quitKey) {
      case 53:
        quit = 1;
        break;

      default:
        break;
      }

      std::array<double, 3> position = getGPSValue();

      if (data.is_open()) {
        data << position[0] << ",";
        data << position[2] << ",";
        data << robots << ",";
        data << getCurrentOrientation() << ",";
        data << " \n";
      }
    }
    move(0, 0);
    data.close();
  }

  void Swarm::randomAlign() {
    int random = randomVal(8);
    while (step(TIME_STEP != -1)) {
      saveCompassValues();
      int test2 = getCurrentHeading();
      if (test2 != random) {
        align(random);
      } else {
        move(0, 0);
        break;
      }
    }
  }

  int Swarm::randomVal(int range) {
    int robNum =
        atoi(robot_name.substr(1, strlen(robot_name.c_str()) - 1).c_str());
    return matrix[robNum][rand() % 100] % range;
  }

  void Swarm::objectDetectionMode() {
    setLEDs(1);
    while (step(TIME_STEP != -1)) {
      objectDetection(1.0);
    }
  }

  void Swarm::teleop() {
    std::string test;
    std::string filename = printName() + "_data_" + test + ".csv";
    std::ofstream data(filename, std::ios::out);

    while (step(TIME_STEP) != -1) {
      if (robot_name.compare("e2") == 0) {
        setLEDs(0);
        // -----get data-----
        int keyPress = readKey();

        saveCompassValues();

        // -----process data-----

        // -----send actuator commands-----
        keyboardControl(keyPress);
        std::array<double, 3> position = getGPSValue();
        std::cout << printName();
        std::cout << "GPS ";
        for (int i = 0; i < 3; i++) {
          std::cout << position[i] << " ";
        }
        std::cout << std::endl;
      }
      if (robot_name.compare("e2") != 0) {
        flock();
        sendCurrentOrientation();
      }
    }
  }

  void Swarm::computeDirections(std::array<double, 100> &allDirections) {
    std::array<double, 3> vector;
    for (int i = 0; i < robots; i++) {
      for (int j = 0; j < 3; j++) {
        vector[j] = emitterDirection[i][j];
      }

      allDirections[i] = computeVectorAngle(vector);
    }
  }

  void Swarm::computeCluster(std::array<double, 100> &allDirections,
                             std::array<int, 4> &output) {

    int countArray[4] = {0, 0, 0, 0};
    // Sorts in ascending order
    std::sort(std::begin(allDirections), std::end(allDirections));

    for (int i = 100 - robots; i < 100; i++) {
      if (allDirections[i] >= 0 && allDirections[i] < 90) {
        countArray[0]++;
      } else if (allDirections[i] >= 90 && allDirections[i] < 180) {
        countArray[1]++;
      } else if (allDirections[i] >= 180 && allDirections[i] < 270) {
        countArray[2]++;
      } else if (allDirections[i] >= 270 && allDirections[i] <= 360) {
        countArray[3]++;
      }
    }

    for (int i = 0; i < 4; i++) {
      output[i] = countArray[i];
    }
  }

  int Swarm::chooseSector(std::array<int, 4> &output) {
    // roulette wheel selection using a fitness function
    int sum = 0;
    std::array<int, 4> outputCopy = output;

    for (int i = 0; i < 4; i++) {
      outputCopy[i] = outputCopy[i] * ROULETTE;
      if (outputCopy[i] == 0) {
        outputCopy[i] = 1;
      }
      sum += outputCopy[i];
    }

    int value = randomVal(sum);
    int i = 0;

    while (value >= 0) {
      value -= outputCopy[i];
      i++;
    }
    i--;

    return i;
  }

  bool Swarm::checkSector(int sector) {
    int myHeading = getCurrentHeading();
    int mySector = 0;

    if ((myHeading >= 0) & (myHeading < 2)) {
      mySector = 1;
    }

    if ((myHeading >= 2) & (myHeading < 4)) {
      mySector = 2;
    }

    if ((myHeading >= 4) & (myHeading < 6)) {
      mySector = 3;
    }

    if ((myHeading >= 6) & (myHeading <= 7)) {
      mySector = 4;
    }

    if (mySector == sector) {
      return true;
    } else {
      return false;
    }
  }

  double Swarm::randomSectorAngle(int sector) {
    double angle = randomVal(90);

    switch (sector) {
    case 0:
      angle += 0;
      break;
    case 1:
      angle += 90;
      break;
    case 2:
      angle += 180;
      break;
    case 3:
      angle += 270;
      break;
    }

    return angle;
  }

  void Swarm::adjust(double angle) {
    double left_speed = WHEEL_SPEED;
    double right_speed = WHEEL_SPEED;

    if (((angle > (360 - ALIGN_ERROR)) & (angle < 360)) |
        ((angle > 0) & (angle < ALIGN_ERROR))) {
      right_speed = left_speed = WHEEL_SPEED;
    } else if ((angle > 270) & (angle < (360 - ALIGN_ERROR))) {
      left_speed =
          (1 + (ALIGN_ERROR + angle - 360) / (90 - ALIGN_ERROR)) * WHEEL_SPEED;
      right_speed = WHEEL_SPEED;
    } else if ((angle >= 180) & (angle < 270)) {
      left_speed = ((angle - 270) / 90) * WHEEL_SPEED;
      right_speed = WHEEL_SPEED;
    } else if ((angle >= 90) & (angle < 180)) {
      right_speed = ((90 - angle) / (90)) * WHEEL_SPEED;
      left_speed = WHEEL_SPEED;
    } else if ((angle >= ALIGN_ERROR) & (angle < 90)) {
      right_speed =
          (1 - (angle - ALIGN_ERROR) / (90 - ALIGN_ERROR)) * WHEEL_SPEED;
      left_speed = WHEEL_SPEED;
    }

    if (left_speed > 1000) {
      left_speed = 1000;
    }

    if (right_speed > 1000) {
      right_speed = 1000;
    }

    move((int)left_speed, (int)right_speed);
  }

  void Swarm::sendCurrentSpeed() {
    std::string message = std::to_string((int)diffWheels->getLeftSpeed()) + " " +
                          std::to_string((int)diffWheels->getRightSpeed());
    sendPacket(message);
  }

  void Swarm::sendCurrentOrientation() {
    std::string message = std::to_string(getCurrentOrientation());
    sendPacket(message);
  }

  void Swarm::getReceiverData() {
    Receiver *copy = (Receiver *)malloc(sizeof(Receiver));
    myData.orientationString = {};
    robots = roundNum((receiver->getQueueLength() + 1) / 2);
    for (int i = 0; i < 100; i++) {
      signalStrength[i] = 0;
    }
    for (int k = 0; k < receiver->getQueueLength(); k++) {
      data = (char *)receiver->getData();
      signalStrength[k] = (double)receiver->getSignalStrength();
      for (int n = 0; n < 3; n++) {
        emitterDirection[k][n] = receiver->getEmitterDirection()[n];
      }
      memcpy(copy, data, sizeof(Receiver));
      myData.orientationString[k] = (char *)copy;
      receiver->nextPacket();
    }
  }

  void Swarm::processReceiverData(std::array<std::string, ARRAY_SIZE> data) {
    myData.received = false;
    for (unsigned int i = 0; i < ARRAY_SIZE; i++) {
      myData.orientationDouble[i] = 0;
    }
    try {
      std::regex re("[*0-9*.*0-9*]+");
      for (int i = 0; i < robots; i++) {
        std::sregex_iterator next(data[i].begin(), data[i].end(), re);
        std::sregex_iterator end;
        while (next != end) {
          std::smatch match = *next;
          std::string match1 = match.str();
          myData.orientationDouble[i] = atoi(match1.c_str());
          if (myData.orientationDouble[i] >= 352) {
            myData.orientationDouble[i] = 0;
          }
          next++;
          if (next == end) {
            myData.received = true;
          }
        }
      }
    } catch (std::regex_error &e) {
    }
  }

  void Swarm::sendPacket(std::string message) {
    std::string temp = printName();
    temp.append(message);
    const char *packet = temp.c_str();
    emitter->send(packet, (strlen(packet) + 1));
  }

  void Swarm::distanceCheck() {
    for (int i = 0; i < 8; i++) {
      ps_values[i] = checkDistanceSensor(i);
    }

    // detect obsctacles
    right_obstacle = (ps_values[0] > PS_THRESHOLD) ||
                     (ps_values[1] > PS_THRESHOLD) ||
                     (ps_values[2] > PS_THRESHOLD);
    left_obstacle = (ps_values[5] > PS_THRESHOLD) ||
                    (ps_values[6] > PS_THRESHOLD) ||
                    (ps_values[7] > PS_THRESHOLD);
    front_obstacle =
        (ps_values[0] > PS_THRESHOLD) & (ps_values[7] > PS_THRESHOLD);
    back_obstacle = (ps_values[3] > PS_THRESHOLD) & (ps_values[4] > PS_THRESHOLD);
  }

  void Swarm::objectDetection(double speedAdjust) {
    distanceCheck();

    double left_speed = speedAdjust * WHEEL_SPEED;
    double right_speed = speedAdjust * WHEEL_SPEED;

    if (!left_obstacle & right_obstacle) {
      left_speed -= speedAdjust * WHEEL_SPEED;
      right_speed += speedAdjust * WHEEL_SPEED;
    } else if (!right_obstacle & left_obstacle) {
      left_speed += speedAdjust * WHEEL_SPEED;
      right_speed -= speedAdjust * WHEEL_SPEED;
    } else if (right_obstacle & left_obstacle) {
      left_speed = -speedAdjust * WHEEL_SPEED;
      right_speed = -speedAdjust * WHEEL_SPEED;
    }

    if (left_speed > 1000) {
      left_speed = 1000;
    }

    if (right_speed > 1000) {
      right_speed = 1000;
    }
    move((int)left_speed, (int)right_speed);
  }

  void Swarm::separation() {
    distanceCheck();

    // init speeds
    double left_speed = WHEEL_SPEED;
    double right_speed = WHEEL_SPEED;

    // modify speeds according to obstacles
    if (left_obstacle | !right_obstacle && !back_obstacle) {
      left_speed = left_speed;
      right_speed = -right_speed;
    } else if (!left_obstacle | right_obstacle && !back_obstacle) {
      left_speed = -left_speed;
      right_speed = right_speed;
    } else if (back_obstacle) {
      left_speed = right_speed = WHEEL_SPEED;
    } else {
      left_speed = right_speed = 0;
    }

    move((int)left_speed, (int)right_speed);
  }

  void Swarm::align(int heading) {

    double target = 0;
    switch (heading) {
    case 0:
      target = 359;
      break;
    case 1:
      target = 45;
      break;
    case 2:
      target = 90;
      break;
    case 3:
      target = 135;
      break;
    case 4:
      target = 180;
      break;
    case 5:
      target = 225;
      break;
    case 6:
      target = 270;
      break;
    case 7:
      target = 315;
      break;
    }
    int current = getCurrentOrientation();
    int max = target + ALIGN_ERROR;
    int min = target - ALIGN_ERROR;
    if (min <= 0) {
      min = 360 - abs(min);
    }

    if ((current >= min) & (current <= max)) {
      move(0, 0);
      aligned = true;
    } else if (current >= max) {
      move(-WHEEL_SPEED, WHEEL_SPEED);
      aligned = false;
    } else if (current <= min) {
      move(WHEEL_SPEED, -WHEEL_SPEED);
      aligned = false;
    }
  }

  int Swarm::getCurrentHeading() {
    int heading = 0;
    int current = getCurrentOrientation();
    if ((current >= (360 - SECTOR_ANGLE / 2)) | (current < (SECTOR_ANGLE / 2))) {
      heading = 0;
    } else if ((current >= SECTOR_ANGLE / 2) & (current < (1.5 * SECTOR_ANGLE))) {
      heading = 1;
    } else if ((current >= (1.5 * SECTOR_ANGLE)) &
               (current < (2.5 * SECTOR_ANGLE))) {
      heading = 2;
    } else if ((current >= (2.5 * SECTOR_ANGLE)) &
               (current < (3.5 * SECTOR_ANGLE))) {
      heading = 3;
    } else if ((current >= (3.5 * SECTOR_ANGLE)) &
               (current < (4.5 * SECTOR_ANGLE))) {
      heading = 4;
    } else if ((current >= (4.5 * SECTOR_ANGLE)) &
               (current < (5.5 * SECTOR_ANGLE))) {
      heading = 5;
    } else if ((current >= (5.5 * SECTOR_ANGLE)) &
               (current < (6.5 * SECTOR_ANGLE))) {
      heading = 6;
    } else if ((current >= (6.5 * SECTOR_ANGLE)) &
               (current < (7.5 * SECTOR_ANGLE))) {
      heading = 7;
    }
    return heading;
  }

  int Swarm::orientationToHeading(int current) {
    int heading = 0;
    if ((current >= (360 - SECTOR_ANGLE / 2)) | (current < (SECTOR_ANGLE / 2))) {
      heading = 0;
    } else if ((current >= SECTOR_ANGLE / 2) & (current < (1.5 * SECTOR_ANGLE))) {
      heading = 1;
    } else if ((current >= (1.5 * SECTOR_ANGLE)) &
               (current < (2.5 * SECTOR_ANGLE))) {
      heading = 2;
    } else if ((current >= (2.5 * SECTOR_ANGLE)) &
               (current < (3.5 * SECTOR_ANGLE))) {
      heading = 3;
    } else if ((current >= (3.5 * SECTOR_ANGLE)) &
               (current < (4.5 * SECTOR_ANGLE))) {
      heading = 4;
    } else if ((current >= (4.5 * SECTOR_ANGLE)) &
               (current < (5.5 * SECTOR_ANGLE))) {
      heading = 5;
    } else if ((current >= (5.5 * SECTOR_ANGLE)) &
               (current < (6.5 * SECTOR_ANGLE))) {
      heading = 6;
    } else if ((current >= (6.5 * SECTOR_ANGLE)) &
               (current < (7.5 * SECTOR_ANGLE))) {
      heading = 7;
    }
    return heading;
  }

  bool Swarm::checkAlignment() {
    bool alignment = false;
    if (getCurrentHeading() == computeAverageHeading()) {
      alignment = true;
    }
    return alignment;
  }

  void Swarm::saveCompassValues() {
    const double *compassVal = compass->getValues();
    for (int i = 0; i < 3; i++) {
      currentOrientation[i] = compassVal[i];
    }
  }

  std::array<double, 3> Swarm::getGPSValue() {
    const double *dataGPS = gps->getValues();
    std::array<double, 3> GPSout;
    for (int i = 0; i < 3; i++) {
      GPSout[i] = dataGPS[i];
    }

    return GPSout;
  }

  double Swarm::getCurrentOrientation() {
    double angle = 360 - computeVectorAngle(currentOrientation);

    return roundNum(angle);
  }

  double Swarm::computeVectorMagnitude(std::array<double, 3> components) {
    double resultant = sqrt(pow(components[0], 2) + pow(components[2], 2));
    return roundNum(resultant);
  }

  double Swarm::computeVectorAngle(std::array<double, 3> components) {
    double x = components[0];
    double z = components[2];
    double angle = 0;
    double pi = 3.141592;

    // Front of robot taken as 0 degrees i.e -z

    if ((x > 0) && (z < 0)) {
      z = -z;
      angle = 90 - atan((z / x)) * 180 / pi;
    } else if ((x > 0) && (z > 0)) {
      angle = 90 + atan((z / x)) * 180 / pi;
    } else if ((x < 0) && (z > 0)) {
      x = -x;
      angle = 180 + (90 - atan((z / x)) * 180 / pi);
    }
    if ((x < 0) && (z < 0)) {
      x = -x;
      z = -z;
      angle = 270 + atan((z / x)) * 180 / pi;
    }

    return roundNum(angle);
  }

  int Swarm::computeAverageHeading() {
    int headings[robots];

    std::array<double, ARRAY_SIZE> copy = myData.orientationDouble;

    if (robots > 0) {
      for (int i = 0; i < robots; i++) {
        if ((copy[i] >= (360 - SECTOR_ANGLE / 2)) |
            (copy[i] < (SECTOR_ANGLE / 2))) {
          headings[i] = 0;
        } else if ((copy[i] >= SECTOR_ANGLE / 2) &
                   (copy[i] < (1.5 * SECTOR_ANGLE))) {
          headings[i] = 1;
        } else if ((copy[i] >= (1.5 * SECTOR_ANGLE)) &
                   (copy[i] < (2.5 * SECTOR_ANGLE))) {
          headings[i] = 2;
        } else if ((copy[i] >= (2.5 * SECTOR_ANGLE)) &
                   (copy[i] < (3.5 * SECTOR_ANGLE))) {
          headings[i] = 3;
        } else if ((copy[i] >= (3.5 * SECTOR_ANGLE)) &
                   (copy[i] < (4.5 * SECTOR_ANGLE))) {
          headings[i] = 4;
        } else if ((copy[i] >= (4.5 * SECTOR_ANGLE)) &
                   (copy[i] < (5.5 * SECTOR_ANGLE))) {
          headings[i] = 5;
        } else if ((copy[i] >= (5.5 * SECTOR_ANGLE)) &
                   (copy[i] < (6.5 * SECTOR_ANGLE))) {
          headings[i] = 6;
        } else if ((copy[i] >= (6.5 * SECTOR_ANGLE)) &
                   (copy[i] < (7.5 * SECTOR_ANGLE))) {
          headings[i] = 7;
        }
      }
    }

    int m, temp, n;
    for (m = 0; m < robots; m++) {
      for (n = 0; n < robots - m; n++) {
        if (headings[n] > headings[n + 1]) {
          temp = headings[n];
          headings[n] = headings[n + 1];
          headings[n + 1] = temp;
        }
      }
    }

    int target = calculateMode(headings, robots);

    return target;
  }

  int Swarm::calculateMode(int array[], int size) {
    int counter = 1;
    int max = 0;
    int mode = array[0];
    int oldMode = mode;
    int oldMode2 = mode;

    for (int i = 0; i < size - 1; i++) {
      if (array[i] == array[i + 1]) {
        counter++;
        if (counter >= max) {
          if (oldMode2 == oldMode) {
            max = counter;
            oldMode = mode;
            mode = array[i];
          } else if ((array[i] != mode) & (array[i] != oldMode)) {
            oldMode2 = array[i];
          }
        }
      } else {
        counter = 1; // reset counter.
      }
    }

    if ((mode != oldMode) & (mode != oldMode2) & (oldMode != oldMode2)) {
      switch (choose) {
      case 0:
        mode = mode;
        break;

      case 1:
        mode = oldMode;
        break;

      case 2:
        mode = oldMode2;
        break;
      }
    }

    return mode;
  }

  void Swarm::keyboardControl(int keyPress) {

    switch (keyPress) {
    // W
    case 87:
      move(multiplier * speed[0], multiplier * speed[0]);
      break;

    // S
    case 83:
      move(-multiplier * speed[0], -multiplier * speed[0]);
      break;

    // A
    case 65:
      move(-multiplier * speed[0], multiplier * speed[0]);
      break;

    // D
    case 68:
      move(multiplier * speed[0], -multiplier * speed[0]);
      break;

    // Q
    case 81:
      move(multiplier * speed[1], multiplier * speed[0]);
      break;

    // E
    case 69:
      move(multiplier * speed[0], multiplier * speed[1]);
      break;

    // UP ARROW
    case 315:
      if (multiplier < 10) {
        multiplier++;
      }
      break;

    // DOWN ARROW
    case 317:
      if (multiplier > 1) {
        multiplier--;
      }
      break;

    default:
      move(multiplier * speed[2], multiplier * speed[2]);
      break;
    }

    std::cout << "Going " << multiplier << " times base speed" << std::endl;
  }

  int Swarm::readKey() { return keyboard->getKey(); }

  void Swarm::move(int left, int right) { setSpeed(left, right); }

  void Swarm::setLEDs(int value) {
    for (int i = 0; i < 8; i++)
      leds[i]->set(value);
  }

  double Swarm::checkDistanceSensor(int n) {
    return distanceSensors[n]->getValue();
  }

  double Swarm::roundNum(double x) { return floor(x * 100 + 0.5) / 100; }

  std::string Swarm::printName() {
    std::string name = "[" + robot_name + "] ";
    return name;
  }

  int main(int argc, char *argv[]) {

    Swarm *robot = new Swarm();
    robot->run();
    delete robot;
    return 0;
  }

\end{lstlisting}
